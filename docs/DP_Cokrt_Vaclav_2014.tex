% options:
% thesis=B bachelor's thesis
% thesis=M master's thesis
% czech thesis in Czech language
% slovak thesis in Slovak language
% english thesis in English language
% hidelinks remove colour boxes around hyperlinks

\documentclass[thesis=B,czech]{FITthesis}[2012/06/26]

\usepackage[utf8]{inputenc} % LaTeX source encoded as UTF-8

\usepackage{graphicx} %graphics files inclusion
% \usepackage{amsmath} %advanced maths
% \usepackage{amssymb} %additional math symbols

\usepackage{dirtree} %directory tree visualisation

% % list of acronyms
% \usepackage[acronym,nonumberlist,toc,numberedsection=autolabel]{glossaries}
% \iflanguage{czech}{\renewcommand*{\acronymname}{Seznam pou{\v z}it{\' y}ch zkratek}}{}
% \makeglossaries

\newcommand{\tg}{\mathop{\mathrm{tg}}} %cesky tangens
\newcommand{\cotg}{\mathop{\mathrm{cotg}}} %cesky cotangens

% % % % % % % % % % % % % % % % % % % % % % % % % % % % % % 
% ODTUD DAL VSE ZMENTE
% % % % % % % % % % % % % % % % % % % % % % % % % % % % % % 

\department{Katedra softwarového inženýrství}
\title{WebGephi - Webové rozhraní pro Gephi}
\authorGN{Václav} %(křestní) jméno (jména) autora
\authorFN{Čokrt} %příjmení autora
\authorWithDegrees{Bc. Václav Čokrt} %jméno autora včetně současných
% akademických titulů
\supervisor{Ing. Jaroslav Kuchař}
\acknowledgements{Doplňte, máte-li komu a za co děkovat. V~opačném případě úplně odstraňte tento příkaz.}
\abstractCS{Cílem této práce je vytvoření aplikace WebGephi, která zpřístupní funkcionalitu aplikace Gephi\cite{gephi} pomocí standardních 
webových technologií. Gephi je opensource desktopová aplikace sloužící k vizualizaci a manipulaci s grafy. WebGehi bude poskytovat jednotné REST\footnote{Representational State Transfer - architektura webového rozhraní (webových služeb)} rozhranní
k funkcionalitě Gephi. Aplikace bude obsahovat také řízení přístupu - správu uživatelů a klientských aplikací. Součástí řešení bude i ukázková klientská 
aplikace - grafická nadstavba demonstrující funkcionalitu WebGephi.}
\abstractEN{TODO the same in english}
\placeForDeclarationOfAuthenticity{V~Praze}
\declarationOfAuthenticityOption{4} %volba Prohlášení (číslo 1-6)
\keywordsCS{Gephi, Gephi toolkit, graf, REST, webové služby, webová aplikace}
\keywordsEN{Gephi, Gephi toolkit, graph, REST, web services, web application}
\newtheorem{definice}{Definice}

\begin{document}

% \newacronym{CVUT}{{\v C}VUT}{{\v C}esk{\' e} vysok{\' e} u{\v c}en{\' i} technick{\' e} v Praze}
% \newacronym{FIT}{FIT}{Fakulta informa{\v c}n{\' i}ch technologi{\' i}}

\begin{introduction}
	\uv{Graf je základním objektem teorie grafů. Jedná se o reprezentaci množiny objektů, u které chceme znázornit, 
	že některé prvky jsou propojeny. Objektům se přiřadí vrcholy a jejich propojení značí hrany mezi nimi. 
	Grafy slouží jako abstrakce mnoha různých problémů. Často se jedná o zjednodušený model nějaké skutečné 
	sítě (například dopravní), který zdůrazňuje topologické vlastnosti objektů (vrcholů) a zanedbává geometrické vlastnosti, například přesnou polohu.\cite{wiki:graf}}
		
	Graf tedy můžeme chápat jako zjednodušený obraz nejaké skutečnosti -
	seznam lidí a vztahů mezi nimi, množina webových stránek a hypertextových odkazů, cokoli, co lze znázornit jako množinu uzlů a hran (vztahů mezi nimi).
	
	Gephi\cite{gephi} je desktopová platforma sloužící k analýze grafů. Umožňuje interaktivně měnit rozložení grafů na základě jejich struktury, vypočítávat metriky (pagerank, shlukovací koeficient, \ldots), 
	filtovat uzly podle jejich vlastností a mnoho dalšího. To vše slouží k tomu, aby uživatel byl schopný v grafu najít skryté závislosti a mohl lépe pochopit (a vizualizovat) strukturu grafu.
	
	Jednou z hlavních výhod Gephi je jeho rozšiřitelnost. Kdokoli může vytvořit svou vlastní funkci k manipulaci s grafem a ve formě pluginu ji přidat do aplikace.
	
	Aplikace WebGephi by měla zachovat tyto vlastnosti a navíc přidat výhody plynoucí ze standardizovaného rozhraní webové aplikace.  
\end{introduction}

\chapter{Cíl práce}
Cílem této práce je vytvořit webovou aplikaci poskytující funkcionalitu Gephi. Hlavní rozhranní této aplikace bude založeno na architektuře REST.

\begin{definice}
REST (Representational State Transfer) – je architektura rozhraní, navržená pro distribuované prostředí. REST navrhnul a popsal v roce 2000 Roy 
Fielding (jeden ze spoluautorů protokolu http) v rámci disertační práce Architectural Styles and the Design of Network-based Software Architectures. 
V kontextu práce je nejzajímavější kapitola 5, ve které Fielding odvozuje principy RESTu na základě známých přístupů k architektuře. 
Rozhraní REST je použitelné pro jednotný a snadný přístup ke zdrojům (resources). Zdrojem mohou být data, stejně jako stavy aplikace 
(pokud je lze popsat konkrétními daty). REST je tedy na rozdíl od známějších XML-RPC či SOAP, orientován datově, nikoli procedurálně. 
Všechny zdroje mají vlastní identifikátor URI a REST definuje čtyři základní metody pro přístup k nim.\cite{wiki:rest}
\end{definice}

WebGephi bude sloužit jen jako poskytovatel služeb pro jiné (klientské) aplikace. Tyto aplikace budou pomocí webových služeb přistupovat k WebGephi (nahrávat grafy, aplikovat na ně funkce, exportovat výsledky, \ldots)
a komunitovat s koncovým uživatelem přes vlastní grafické rozhraní. Klienské aplikace mohou být webové, desktopové i mobilní aplikace. Přístup klienských aplikací samořejmě bude také třeba řídit. 
WebGephi tedy kromě REST rozhranní bude poskytovat také grafické rozhranní pro registraci a správu uživatelů a klientských aplikací.

Hlavní výhody WebGephi budou jednoduché, standardizované, deklarativní rozhranní. Gephi sice poskytuje knihovnu pro práci s grafy, \uv{Gephi Toolkit}\cite{gephi:toolkit}, ta je však dosti složitá na použití. 
Klientské aplikace budou moci jednoduše přistupovat k tomuto rozhraní pomocí HTTP protokolu, na kterém je v naprosté většině založena REST archtektura. To umožní jak
jednoduché prozkoumávání rozhraní (např. pomocí webového prohlížeče), tak strojové zpracování koncovými klienskými aplikacemi.

Pro koncového uživatele je hlavní výhodou možnost použití bez nutnosti instalace (ve spolupráci s klienskými aplikacemi), použití jako centrálního úložistě grafů a možnost sdílení s jinými uživateli.

WebGephi bude postaveno nad již zmíněnou knihovnou \uv{Gephi Toolkit}. Ta obsahuje základní moduly Gephi (bez GUI\footnote{Graphic User Interface - grafické uživatelské rozhranní} fukcionality) ve formě jednoduché Java knihovny.
V prvním kroku bude třeba určit, jaká funcionalita bude implementována ve WebGephi. Následně bude třeba analyzovat strukturu knihovnu \uv{Gephi Toolkit} a s její pomocí tuto funkcionalitu implementovat.
Dále bude potřeba navrhnout a implementovat strukturu REST rozhraní a způsob autentizace uživatelů.

Součástí práce je i implementace ukázkové klientské aplikace. Ta bude využívat naprostou většinu funkcionality WebGehi a demonstrovat její funčnost.

\chapter{Analýza}
\section{Gephi}

\section{Určení komponent relevatních pro WebGephi}

\section{Gephi Toolkit}

\section{Možnosti autorizace}

\chapter{Návrh}

\section{REST rozhranní}

\section{Autorizace}

\section{Správa uživatelů a klientských aplikací}

\section{Struktura aplikace}
\subsection{Server}
\subsection{Klient (java konektor)}
\subsection{Klientská aplikace}

\section{GUI}

\chapter{Realizace}

\section{Požité technologie}
\subsection{Wildlfy 8}
\subsection{RestEasy}
\subsection{Enterprise Java Beans}
\subsection{JPA}
\subsection{Errai a GWT}
\subsection{Vaadin}

\section{Server TODO popis struktury a zajimavých částí}
\section{Klient (konektor) TODO popis struktury a zajimavých částí}
\section{Klientská aplikace (konektor) TODO popis struktury a zajimavých částí}

\begin{conclusion}
	%sem napište závěr Vaší práce
\end{conclusion}

\bibliographystyle{csn690}
\bibliography{mybibliographyfile}

\appendix


\chapter{Seznam použitých zkratek}
% \printglossaries
\begin{description}
	\item[GUI] Graphical user interface
	\item[XML] Extensible markup language
\end{description}


% % % % % % % % % % % % % % % % % % % % % % % % % % % % 
% % Tuto kapitolu z výsledné práce ODSTRAŇTE.
% % % % % % % % % % % % % % % % % % % % % % % % % % % % 
% 
% \chapter{Návod k~použití této šablony}
% 
% Tento dokument slouží jako základ pro napsání závěrečné práce na Fakultě informačních technologií ČVUT v~Praze.
% 
% \section{Výběr základu}
% 
% Vyberte si šablonu podle druhu práce (bakalářská, diplomová), jazyka (čeština, angličtina) a kódování (ASCII, \mbox{UTF-8}, \mbox{ISO-8859-2} neboli latin2 a nebo \mbox{Windows-1250}). 
% 
% V~české variantě naleznete šablony v~souborech pojmenovaných ve formátu práce\_kódování.tex. Typ může být:
% \begin{description}
% 	\item[BP] bakalářská práce,
% 	\item[DP] diplomová (magisterská) práce.
% \end{description}
% Kódování, ve kterém chcete psát, může být:
% \begin{description}
% 	\item[UTF-8] kódování Unicode,
% 	\item[ISO-8859-2] latin2,
% 	\item[Windows-1250] znaková sada 1250 Windows.
% \end{description}
% V~případě nejistoty ohledně kódování doporučujeme následující postup:
% \begin{enumerate}
% 	\item Otevřete šablony pro kódování UTF-8 v~editoru prostého textu, který chcete pro psaní práce použít -- pokud můžete texty s~diakritikou normálně přečíst, použijte tuto šablonu.
% 	\item V~opačném případě postupujte dále podle toho, jaký operační systém používáte:
% 	\begin{itemize}
% 		\item v~případě Windows použijte šablonu pro kódování \mbox{Windows-1250},
% 		\item jinak zkuste použít šablonu pro kódování \mbox{ISO-8859-2}.
% 	\end{itemize}
% \end{enumerate}
% 
% 
% V~anglické variantě jsou šablony pojmenované podle typu práce, možnosti jsou:
% \begin{description}
% 	\item[bachelors] bakalářská práce,
% 	\item[masters] diplomová (magisterská) práce.
% \end{description}
% 
% \section{Použití šablony}
% 
% Šablona je určena pro zpracování systémem \LaTeXe{}. Text je možné psát v~textovém editoru jako prostý text, lze však také využít specializovaný editor pro \LaTeX{}, např. Kile.
% 
% Pro získání tisknutelného výstupu z~takto vytvořeného souboru použijte příkaz \verb|pdflatex|, kterému předáte cestu k~souboru jako parametr. Vhodný editor pro \LaTeX{} toto udělá za Vás. \verb|pdfcslatex| ani \verb|cslatex| \emph{nebudou} s~těmito šablonami fungovat.
% 
% Více informací o~použití systému \LaTeX{} najdete např. v~\cite{wikilatex}.
% 
% \subsection{Typografie}
% 
% Při psaní dodržujte typografické konvence zvoleného jazyka. České \uv{uvozovky} zapisujte použitím příkazu \verb|\uv|, kterému v~parametru předáte text, jenž má být v~uvozovkách. Anglické otevírací uvozovky se v~\LaTeX{}u zadávají jako dva zpětné apostrofy, uzavírací uvozovky jako dva apostrofy. Často chybně uváděný symbol "{} (palce) nemá s~uvozovkami nic společného.
% 
% Dále je třeba zabránit zalomení řádky mezi některými slovy, v~češtině např. za jednopísmennými předložkami a spojkami (vyjma \uv{a}). To docílíte vložením pružné nezalomitelné mezery -- znakem \texttt{\textasciitilde}. V~tomto případě to není třeba dělat ručně, lze použít program \verb|vlna|.
% 
% Více o~typografii viz \cite{kobltypo}.
% 
% \subsection{Obrázky}
% 
% Pro umožnění vkládání obrázků je vhodné použít balíček \verb|graphicx|, samotné vložení se provede příkazem \verb|\includegraphics|. Takto je možné vkládat obrázky ve formátu PDF, PNG a JPEG jestliže používáte pdf\LaTeX{} nebo ve formátu EPS jestliže používáte \LaTeX{}. Doporučujeme preferovat vektorové obrázky před rastrovými (vyjma fotografií).
% 
% \subsubsection{Získání vhodného formátu}
% 
% Pro získání vektorových formátů PDF nebo EPS z~jiných lze použít některý z~vektorových grafických editorů. Pro převod rastrového obrázku na vektorový lze použít rasterizaci, kterou mnohé editory zvládají (např. Inkscape). Pro konverze lze použít též nástroje pro dávkové zpracování běžně dodávané s~\LaTeX{}em, např. \verb|epstopdf|.
% 
% \subsubsection{Plovoucí prostředí}
% 
% Příkazem \verb|\includegraphics| lze obrázky vkládat přímo, doporučujeme však použít plovoucí prostředí, konkrétně \verb|figure|. Například obrázek \ref{fig:float} byl vložen tímto způsobem. Vůbec přitom nevadí, když je obrázek umístěn jinde, než bylo původně zamýšleno -- je tomu tak hlavně kvůli dodržení typografických konvencí. Namísto vynucování konkrétní pozice obrázku doporučujeme používat odkazování z~textu (dvojice příkazů \verb|\label| a \verb|\ref|).
% 
% \begin{figure}\centering
% 	\includegraphics[width=0.5\textwidth, angle=30]{cvut-logo-bw}
% 	\caption[Příklad obrázku]{Ukázkový obrázek v~plovoucím prostředí}\label{fig:float}
% \end{figure}
% 
% \subsubsection{Verze obrázků}
% 
% % Gnuplot BW i barevně
% Může se hodit mít více verzí stejného obrázku, např. pro barevný či černobílý tisk a nebo pro prezentaci. S~pomocí některých nástrojů na generování grafiky je to snadné.
% 
% Máte-li například graf vytvořený v programu Gnuplot, můžete jeho černobílou variantu (viz obr. \ref{fig:gnuplot-bw}) vytvořit parametrem \verb|monochrome dashed| příkazu \verb|set term|. Barevnou variantu (viz obr. \ref{fig:gnuplot-col}) vhodnou na prezentace lze vytvořit parametrem \verb|colour solid|.
% 
% \begin{figure}\centering
% 	\includegraphics{gnuplot-bw}
% 	\caption{Černobílá varianta obrázku generovaného programem Gnuplot}\label{fig:gnuplot-bw}
% \end{figure}
% 
% \begin{figure}\centering
% 	\includegraphics{gnuplot-col}
% 	\caption{Barevná varianta obrázku generovaného programem Gnuplot}\label{fig:gnuplot-col}
% \end{figure}
% 
% 
% \subsection{Tabulky}
% 
% Tabulky lze zadávat různě, např. v~prostředí \verb|tabular|, avšak pro jejich vkládání platí to samé, co pro obrázky -- použijte plovoucí prostředí, v~tomto případě \verb|table|. Například tabulka \ref{tab:matematika} byla vložena tímto způsobem.
% 
% \begin{table}\centering
% 	\caption[Příklad tabulky]{Zadávání matematiky}\label{tab:matematika}
% 	\begin{tabular}{|l|l|c|c|}\hline
% 		Typ		& Prostředí		& \LaTeX{}ovská zkratka	& \TeX{}ovská zkratka	\tabularnewline \hline \hline
% 		Text		& \verb|math|		& \verb|\(...\)|	& \verb|$...$|		\tabularnewline \hline
% 		Displayed	& \verb|displaymath|	& \verb|\[...\]|	& \verb|$$...$$|	\tabularnewline \hline
% 	\end{tabular}
% \end{table}
% 
% % % % % % % % % % % % % % % % % % % % % % % % % % % % 

\chapter{Obsah přiloženého CD}

%upravte podle skutecnosti

\begin{figure}
	\dirtree{%
		.1 readme.txt\DTcomment{stručný popis obsahu CD}.
		.1 exe\DTcomment{adresář se spustitelnou formou implementace}.
		.1 src.
		.2 impl\DTcomment{zdrojové kódy implementace}.
		.2 thesis\DTcomment{zdrojová forma práce ve formátu \LaTeX{}}.
		.1 text\DTcomment{text práce}.
		.2 thesis.pdf\DTcomment{text práce ve formátu PDF}.
		.2 thesis.ps\DTcomment{text práce ve formátu PS}.
	}
\end{figure}

\end{document}
